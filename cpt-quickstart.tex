\chapter{Getting Started}
\label{sct:start}

This section describes how to install the \cvmfs\ client.
\cvmfs\ is supported on Scientific Linux 5 and 6, Ubuntu 12.04, openSuSE 12.2, Fedora 17, and Mac OS X.

\section{Getting the Software}
The \cvmfs\ source code and binary packages are available under \url{https://cernvm.cern.ch/portal/downloads}.
Binary packages are produced for \rpm, \dpkg, and Mac OS X (.pkg).
\yum\ repositories for \SI{64}{\bit} and \SI{32}{\bit} Scientific Linux 5 and 6 are available under \url{http://cvmrepo.web.cern.ch/cvmrepo/yum}.
The \texttt{cvmfs-release} packages can be used to add a these \yum\ repositories to the local \yum\ installation.
The \texttt{cvmfs-release} packages are available under \url{https://cernvm.cern.ch/portal/downloads}.

The \cvmfs\ client is not relocatable and needs to be installed under /usr.
In order to compile and install from sources, use the following \cmake\ command:
\begin{verbatim}
  cmake .
  make
  sudo make install
\end{verbatim}

\section{Installation}
\subsection{Linux}
To install, proceed according to the following steps:
\begin{description}
	\item[Step 1] Install the \cvmfs\ packages.  With yum, run
\begin{verbatim}
  yum install cvmfs-keys cvmfs cvmfs-init-scripts.
\end{verbatim}
			      If yum does not show the latest packages, clean the yum cache by \texttt{yum clean all}.
			      Use \texttt{rpm -vi} to install the packages using just \rpm.
			      On Ubuntu, use \texttt{dpkg -i} on the cvmfs .deb package.
    \item[Step 2] For the base setup, run \texttt{cvmfs\_config setup}.
    				Alternatively, you can do the base setup by hand: ensure that \texttt{user\_allow\_other} is set in /etc/fuse.conf and ensure that \texttt{/cvmfs /etc/auto.cvmfs} is set in /etc/auto.master.
					If you migrate from a previous version of \cvmfs, check the release notes if there is anything special to do for migration.
	\item[Step 3] Create /etc/cvmfs/default.local and open the file for editing.
	\item[Step 4] Select the desired repositories by setting \texttt{CVMFS\_REPOSITORIES=repo1,repo2,...}.
		For ATLAS, for instance, set 
\begin{verbatim}
  CVMFS_REPOSITORIES=atlas.cern.ch,atlas-condb.cern.ch,grid.cern.ch
\end{verbatim}
		Specify the HTTP proxy servers on your site with
\begin{verbatim}
  CVMFS_HTTP_PROXY="http://myproxy1:port|http://myproxy2:port"
\end{verbatim}
		For the syntax of more complex HTTP proxy settings, see Section~\ref{sct:config:network}.
		Make sure your local Squid servers allow access to all the Stratum 1 web servers~\ref{sct:squid}.
		For Cern repositories, the Stratum 1 web servers are listed in /etc/cvmfs/domain.d/cern.ch.conf.
	\item[Step 5] Check if \cvmfs\ mounts the specified repositories by \texttt{cvmfs\_config probe}.
\end{description}

\subsection{Mac OS X}
On Mac OS X, \cvmfs\ is based on \fusex\footnote{\url{http://fuse4x.github.com}}. 
It is not yet integrated with \autofs.
In order to install, proceed according to the following steps:
\begin{description}
	\item[Step 1] Install the \cvmfs\ package by opening the .pkg file.
	\item[Step 2] Create /etc/cvmfs/default.local and open the file for editing.
	\item[Step 3] Select the desired repositories by setting \texttt{CVMFS\_REPOSITORIES=repo1,repo2,...}.
		For CMS, for instance, set 
\begin{verbatim}
  CVMFS_REPOSITORIES=cms.cern.ch
\end{verbatim}
		Specify the HTTP proxy servers on your site with
\begin{verbatim}
  CVMFS_HTTP_PROXY="http://myproxy1:port|http://myproxy2:port"
\end{verbatim}
	If you're unsure about the proxy names, set \texttt{CVMFS\_HTTP\_PROXY=DIRECT}.
    \item[Step 4] Mount your repositories like
\begin{verbatim} 
  sudo mkdir -p /cvmfs/cms.cern.ch
  sudo mount -t cvmfs cms.cern.ch /cvmfs/cms.cern.ch
\end{verbatim}
\end{description}

\section{Usage}
The \cvmfs\ repositories are located under /cvmfs.
Each repository is identified by a \emph{fully qualified repository name}.
The fully qualified repository name consists of a repository identifier and a domain name, similar to DNS records~\cite{rfc1035}.
The domain part of the fully qualified repository name indicates the location of repository creation and maintenance.
For the ATLAS experiment software, for instance, the fully qualified repository name is atlas.cern.ch although the hosting web servers are spread around the world.

Mounting and un-mounting of the \cvmfs\ is controlled by \autofs\ and \product{automount}.
That is, starting from the base directory /cvmfs different repositories are mounted automatically just by accessing them.
For instance, running the command \texttt{ls /cvmfs/atlas.cern.ch} will mount the ATLAS software repository.
This directory gets automatically unmounted after some \product{automount}-defined idle time.

\section{Debugging Hints}
\label{sct:debugginghints}

In order to check for common misconfigurations in the base setup, run
\begin{verbatim}
  cvmfs_config chksetup
\end{verbatim}

\cvmfs\ gathers its configuration parameter from various configuration files that can overwrite each others settings (default configuration, domain specific configuration, local setup, \dots).  
To show the effective configuration for \emph{repository}.cern.ch, run
\begin{verbatim}
  cvmfs_config showconfig repository.cern.ch
\end{verbatim}

In order to exclude \product{autofs}/\product{automounter} as a source of problems, you can try to mount \emph{repository}.cern.ch manually by
\begin{verbatim}
  mkdir -p /mnt/cvmfs
  mount -t cvmfs repository.cern.ch /mnt/cvmfs
\end{verbatim}

In order to exclude SELinux as a source of problems, you can try mounting after SELinux has been disabled by
\begin{verbatim}
  /usr/sbin/setenforce 0
\end{verbatim}

Once you sorted out a problem, make sure that you do not get the original error served from the file system buffers by
\begin{verbatim}
  service autofs restart
\end{verbatim}

In case you need additional assistance, please don't hesitate to contact us at \url{cernvm.support@cern.ch}.
Together with the problem description, please send the system information tarball created by \texttt{cvmfs\_config bugreport}.
