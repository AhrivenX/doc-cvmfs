
% This and that
%------------------
\usepackage{etex} % use the etex engine, more variables etc.
\usepackage[utf8]{inputenc}
\usepackage{cmap}    % Correct encoding for Umlaute in pdf
\usepackage[T1]{fontenc}    % Correct hyphenation for non-ASCII characters
\usepackage[english]{babel}
\usepackage{calc}
\usepackage[left]{eurosym}
\usepackage{thumbpdf}
\usepackage{cite} % Order citations automatically (such as [1-4,5,8])
\usepackage{array} % Extend array and tabular with column formatting
\usepackage{hyperref}
\hypersetup{final,
	        plainpages=false,
	        pdfpagelabels,
	        pagebackref,
	        %hyperfootnotes=false,
	        colorlinks=true,
	        linkcolor=blue,
	        urlcolor=red,
	        citecolor=blue,
	        pdfpagemode=UseOutlines,
	        pdfstartview=FitH,
	        pdfborder={0 0 0}}
\usepackage{url}
\usepackage[per-mode=symbol,bracket-unit-denominator=false,detect-all,load-configurations=binary]{siunitx}
\DeclareSIUnit\loc{LOC}
\DeclareSIUnit\permil{\textperthousand}

\providecommand{\todo}[1]{{\Large TODO: #1}}
\providecommand{\newterm}[1]{\emph{#1}}
\providecommand{\indexed}[1]{\index{#1}#1}
\providecommand{\product}[1]{{\scshape #1}\index{#1@\textsc{#1}}}
\newcommand{\HRule}{\rule{\linewidth}{0.5mm}}
\newcommand{\ie}{i.\,e.\ }
\newcommand{\eg}{e.\,g.\ }
\newcommand{\cf}{cf.\ }
\newcommand{\dotfillbox}[1]{\makebox[#1]{\dotfill}}


% Tables
%-------------
\usepackage{tabularx}

% Split a cell into multiple rows/columns 
\usepackage{multicol}
\usepackage{multirow}

%\usepackage{slashbox} % Diagonal cells
\usepackage{booktabs}
\usepackage{rotating}


% Graphics
%--------------
\usepackage[all]{xy}
\usepackage{xcolor}
\definecolor{light}{gray}{0.8}
\definecolor{tuerkis}{cmyk}{0.5,0.15,0,0.3}
\usepackage{graphicx}
\usepackage{tikz}
\usetikzlibrary{arrows,positioning,shapes,topaths,calc,fit,backgrounds,matrix,shadows,automata,patterns,decorations.pathmorphing,decorations.pathreplacing,decorations.text,circuits.logic.US,trees,mindmap}
\usepackage{gnuplot-lua-tikz} % GNUplot plots


% Math
%--------
\usepackage{amsmath,amssymb,amstext,amsthm,amsfonts}
%\usepackage{dsfont} %Font for nice set symbols (R, N, Z, ...)
\usepackage{nicefrac} %"Marktfrauenbruch"
% Set symbols
\newcommand{\R}{\ensuremath{\mathds{R}}}
\renewcommand{\P}{\ensuremath{\mathds{P}}}
\newcommand{\N}{\ensuremath{\mathds{N}}}
\newcommand{\Z}{\ensuremath{\mathds{Z}}}
\newcommand{\Q}{\ensuremath{\mathds{Q}}}
\newcommand{\F}{\ensuremath{\mathds{F}}}
\newcommand{\C}{\ensuremath{\mathds{C}}}
% Operators
\newcommand{\entspricht}{\ensuremath{\mathrel{\widehat{=}}}} % entspricht
\providecommand{\abs}[1]{\left\lvert #1 \right\rvert} % Betrag
\providecommand{\norm}[1]{\left\lVert #1 \right\rVert} % Norm
\providecommand{\floor}[1]{\left\lfloor #1 \right\rfloor}   
\providecommand{\ceil}[1]{\left\lceil #1 \right\rceil}   
\providecommand{\svert}{\; \vert \;} %Single vertical bar
\DeclareMathOperator{\grad}{grad} % Gradient
\DeclareMathOperator{\rot}{rot} % Rotation
\DeclareMathOperator{\Div}{div} % Divergenz
\DeclareMathOperator{\rg}{rg} % Rang
\DeclareMathOperator{\Grad}{Grad} % Grad (eines Polynoms)
\DeclareMathOperator{\Abb}{Abb} % Abbildungsmenge
\DeclareMathOperator{\Sym}{Sym} % Bijektionenmenge
\DeclareMathOperator{\Hom}{Hom} % Homomorphismenmenge
\DeclareMathOperator{\End}{End} % Endomorphismenmenge
\DeclareMathOperator{\Aut}{Aut} % Automorphismenmenge
\DeclareMathOperator{\deF}{def} 
\DeclareMathOperator{\ran}{ran} 
\DeclareMathOperator{\dist}{d}
\DeclareMathOperator{\rx}{rx}
\DeclareMathOperator{\tx}{tx} 
\DeclareMathOperator{\req}{req}
\DeclareMathOperator{\avg}{avg}
\DeclareMathOperator{\vol}{vol}
\DeclareMathOperator{\identical}{id} 
\providecommand{\id}{\ensuremath{\textsf{id}}} % identische Abbildung
\DeclareMathOperator{\prob}{\mathbf{P}} % Probablility
\DeclareMathOperator{\E}{\mathbf{E}} % Expected value
\DeclareMathOperator{\percentile}{P} % Percentil
\DeclareMathOperator{\definedas}{\mathrel{\mathop:}=}

% Arbitrarily long double arrows
\makeatletter
\def\Relbar{\mathrel{\smash=}}
\def\Leftarrowfill@{\arrowfill@\Leftarrow\Relbar\Relbar}
\def\Rightarrowfill@{\arrowfill@\Relbar\Relbar\Rightarrow}
\newcommand{\xRightarrow}[2][]{\ext@arrow 0359\Rightarrowfill@{#1}{#2}}
\newcommand{\xLeftarrow}[2][]{\ext@arrow 3095\Leftarrowfill@{#1}{#2}}
\makeatother 


% Informatics
%----------------
\usepackage[ruled, vlined, algo2e]{algorithm2e}

\usepackage[final]{listings}
\lstset{numbers=left, numberstyle=\tiny, stepnumber=2, numbersep=5pt, frame=lines, basicstyle=\small, language=c}

\newcommand{\np}{\ensuremath{\mathcal{NP}}}
\renewcommand{\O}{\ensuremath{\mathcal{O}}}


% Typography
%------------------
\usepackage{microtype}
\usepackage{slantsc} % Combine sc fonts with italic, bold, ...
%\usepackage{geometry} % Page dimensions
\clubpenalty = 9000 % No Schusterjungen
\widowpenalty = 9000 \displaywidowpenalty = 9000 % No Hurenkinder
% Roman numbers
\newcommand{\romannum}[1]{
	\ifnum#1<1 
		\ifnum#1=0 
			o
		\else 
			-\romannumeral -#1
		\fi 
	\else 
		\romannumeral #1
	\fi}
\DeclareRobustCommand{\Romannum}[1]{\MakeUppercase{\romannum{#1}}}	
