%\begin{center}
%\begin{longtable}
\begin{longtable}{lX}
	\toprule
	{\bf\centering File Path} & {\bf\centering Description} \\
	\midrule
	\texttt{/srv/cvmfs} & \textbf{Central repository storage location} \newline
	Can be mounted or symlinked to somewhere else \emph{before} creating the first repository. \\
	\addlinespace

	\texttt{/srv/cvmfs/<fqrn>} & \textbf{Storage location of a specfic repository} \newline
	Can be symlinked to somewhere else \emph{before} creating the repository \texttt{<fqrn>}. \\
	\addlinespace

	\texttt{/var/spool/cvmfs} & \textbf{Internal states of the repositories} \newline
	Can be mounted or symlinked to somewhere else \emph{before} creating the first repository. Hosts the scratch area described in Section~\ref{sct:repoupdate} thus might consume notable disk space during repository updates. \\			\addlinespace

	\texttt{/etc/cvmfs} & \textbf{Configuration files and keychains} \newline
	Similar to the structure described in Table~\ref{tbl:configfiles}. Please don't symlink this directory. \\
	\addlinespace

	\texttt{/etc/cvmfs/cvmfs\_server\_hooks.sh} & \textbf{Customisable server behaviour} \newline
	See Section~\ref{sct:serverhooks} for further details. \\
	\addlinespace

	\texttt{/etc/cvmfs/repositories.d} & \textbf{Repository configuration location} \newline
	Contains repository server specific configuration files.
	 \\
	\bottomrule
	\caption{List of configuration files and directories for the \cvmfs\ server installation}
	\label{tbl:serveranatomyelements}
\end{longtable}
%\end{longtable}
%\end{center}
