\chapter{Setting up a Local Squid Proxy}
\label{sct:squid}

For clusters of nodes with \cvmfs\ clients, we strongly recommend to setup two or more \squid\footnote{\url{http://www.squid-cache.org}} forward proxy servers as well.
The forward proxies will reduce the latency for the local worker nodes, which is critical for cold cache performance. 
They also reduce the load on the Stratum 1 servers.

From what we have seen, a \squid\ server on commodity hardware scales well for at least a couple of hundred worker nodes.
The more RAM and hard disk you can devote for caching the better.
We have good experience with \SIrange{4}{8}{\giga\byte} of memory cache and \SIrange{50}{100}{\giga\byte} of hard disk cache.
We suggest to setup two identical \squid\ servers for reliability and load-balancing.
Assuming the two servers are A and B, set
\begin{verbatim}
  CVMFS_HTTP_PROXY="http://A:3128|http://B:3128"
\end{verbatim}

\squid\ is very powerful and has lots of configuration and tuning options.
For \cvmfs\ we require only the very basic static content caching.
If you already have a \emph{Frontier Squid}\footnote{\url{http://frontier.cern.ch}}~\cite{frontier08, frontier10} installed you can use it as well for \cvmfs.

Otherwise, cache sizes and access control needs to be configured in order to use the \squid\ server with \cvmfs.
In order to do so, browse through your /etc/squid/squid.conf and make sure the following lines appear accordingly:
\begin{verbatim}
  max_filedesc 8192
  maximum_object_size 1024 MB

  cache_mem 128 MB
  maximum_object_size_in_memory 128 KB
  # 50 GB disk cache
  cache_dir ufs /var/spool/squid 50000 16 256
\end{verbatim}

Furthermore, \squid\ needs to allow access to all Stratum 1 servers.
This is controlled through \squid\ ACLs.
For the Stratum 1 servers for the cern.ch, egi.eu, and opensciencegrid.org domains, add the following lines to you \squid\ configuration:
\begin{verbatim}
  acl cvmfs dst cvmfs-stratum-one.cern.ch
  acl cvmfs dst cernvmfs.gridpp.rl.ac.uk
  acl cvmfs dst cvmfs.racf.bnl.gov
  acl cvmfs dst cvmfs02.grid.sinica.edu.tw
  acl cvmfs dst cvmfs.fnal.gov
  acl cvmfs dst cvmfs-atlas-nightlies.cern.ch
  acl cvmfs dst cvmfs-egi.gridpp.rl.ac.uk
  acl cvmfs dst klei.nikhef.nl
  acl cvmfs dst cvmfsrepo.lcg.triumf.ca
  acl cvmfs dst cvmfsrep.grid.sinica.edu.tw
  acl cvmfs dst cvmfs-s1bnl.opensciencegrid.org
  acl cvmfs dst cvmfs-s1fnal.opensciencegrid.org
  http_access allow cvmfs
\end{verbatim}

The \squid\ configuration can be verified by \texttt{squid -k parse}.
Before the first service start, the cache space on the hard disk needs to be prepared by \texttt{squid -z}.
In order to make the increased number of file descriptors effective for Squid, execute \texttt{ulimit -n 8192} prior to starting the squid service.
